% Preamble
% ---
\documentclass{article}

% Packages
% ---
% Adjust margins and paper size/orientation here
\usepackage[letterpaper, portrait, margin=.5in]{geometry}
\usepackage{amsmath} % Advanced math typesetting
\usepackage[T1]{fontenc} % Font encoding for french
\usepackage[utf8]{inputenc} % Unicode support (Umlauts etc.)
\usepackage[french]{babel} % Change hyphenation rules
\usepackage{hyperref} % Add a link to your document
\usepackage{graphicx} % Add pictures to your document
\usepackage{listings} % Source code formatting and highlighting
\usepackage[document]{ragged2e}
\usepackage{enumitem}
\begin{document}
\subsection*{Définitions}

\begin{description}
\item [predictors:] Variables d'entrées/\emph{input variables} utilisées dans le
  cadre d'un problème d'apprentissage supervisé.
\end{description}

\subsection*{Interrogations/Brainstorm}
\textbf{TOUTES LES NOTIONS CI-DESSOUS RESTENT A COMPRENDRE OU VALIDER!!!}

\begin{enumerate}
\item  \textbf{Inspection / Visualisation des données}:
  \begin{itemize}[itemsep=1em]
  \item Afin de minimiser le biais du surapprentissage, devrait-on réserver un \emph{test set}
    avant même de regarder les données?
    
    Le simple fait de visualiser les données nous portera à faire des choix.
    Ces choix intégrés dans notre processus, sont eux-mêmes une forme
    d'apprentissage.
    
    D'un autre côté,  si on veut un créer un \emph{stratified test set},
    représentatif de la population futur, il faut faire une analyse minimum
    des données. Et même ce \emph{stratified test set}, s'il est conçu à partir d'un
    sous-ensemble sélectionné de \emph{predictors}, il y a une assomption dans
    cette sélection. C'est ce que \emph{Géron} fait dans le chapître 2 en
    créant sont \emph{test set} à l'aide d'une catégorie de revenue.

  \item Est-ce plus pertinent dans certains cas d'avoir un test set non
    stratifié, mais qui couvre mieux l'ensemble des profils d'entrées, par
    exemple en surreprésentant certains outliers?

    \emph{e.g.} Comment mon modèle généralise-t-il pour les cas plus problématique?

  \item Afin de minimiser le biais du surapprentissage, devrait-on réserver un \emph{test set}
    avant d'effectuer une préselection des \emph{predictors}, ou encore
    l'étape du \emph{data cleanup}.

    Il y des cas par exemple où si le nombre de \emph{predictors} dépassent
    largement le nombre d'échantillons du \emph{dataset}, il est impératif
    de former son \emph{test set} avant de faire une sélection de \emph{predictors}.

    Voir: \url{https://youtu.be/S06JpVoNaA0}
  \end{itemize}
\item \textbf{Étape de nettoyage}:
  \begin{itemize}[itemsep=1em]
  \item \textbf{Donn\'ees manquantes / valeurs nan}:

    La plupart des modèles mathématiques ne gèrent pas bien les donn\'ees
    manquantes. Il faut nettoyer les données pour utiliser ces modèles
    durant l'étape d'apprentissage. Par contre, il ne faut jamais perdre de
    vue que ce même modèle utilisé en production pourrait aussi recevoir
    des entrées incomplètes. Il faut établir une stratégie de nettoyage de
    données autant pour la étape d'apprentissage que la étape de
    production.

    Il est donc très important de comprendre en quoi cette stratégie de
    nettoyage de données affecte le processus d'apprentissage. Comment
    peut-elle biaiser les résultats?

    Par exemples:
    \begin{itemize}
    \item Est-ce que la distribution des échantillons ayant des donn\'ees
      manquantes sera la m\^eme dans le futur?
    \item Est-ce qu'une donnée manquante est une information pertinente en soi?

      Exemple: Dans le jeu de données du Titanic, il semble que le fait qu'un
      passager n'ait pas de num\'ero de cabine (\emph{nan}), pourrait indiquer
      qu'il n'ait pas de cabine qui lui ait \'et\'e attribu\'ee ou r\'eserv\'ee.
      cabine. Peu importe la raison, la donnée manquante a une certaine valeur
      pr\'edictive en soi.
    \end{itemize}

  \item \textbf{Strat\'egies de nettoyage:}
    \begin{enumerate}
    \item \underline{\'Elimination de données/\emph{row}}: Pour les colonnes où il y a peu
      de \emph{nan} (\emph{e.g.} 1\% \`a 2\% du jeu de données), on peut
      sans trop de risques éliminer les \'echantillons/lignes incluant ces \emph{nan}.

    \item \underline{\'Elimination du \emph{predictor}/\emph{colonne}}: Si un
      \emph{predictor} ne semble pas pertinent, ou qu'il manque trop de données,
      on peut l'éliminer.

    \item \underline{Strat\'egie de la moyenne/m\'ediane/mode/``valeur neutre''}: On remplace
      les données manquantes d'une colonne, par la
      moyenne/m\'ediane/mode/``valeur neutre''.

      Si ces données sont en quantit\'es non-n\'egligeagles, cela vient
      modifier la distribution de la colonne modifi\'ee.

      Dans quelles circonstances corrompt-on l'apprentissage, lorsqu'on floue le modèle
      lorsqu'on lui indique que pour un \'echantillon donnée, un de ces
      \emph{predictors} est égal à la moyenne/m\'ediane/mode, alors qu'en
      r\'ealit\'e on ne conna\^it pas la vraie valeur?

    \item \underline{Ajouter un label \emph{unknown} si le \emph{predictor} est
        une catégorie}: Pour une \emph{predictor} de type catégorie, on peut
      simplement ajouter un label \emph{unknown}, pour indiquer au modèle qu'on
      ne connaît la valeur.

    \item \underline{Transformation d'un \emph{predictor} $\in{\mathbb{R}}$ en une
      catégorie}: On transforme notre \emph{predictor} en catégorie et on
    applique la stratégie pr\'ec\'edente.

    \end{enumerate}
  \end{itemize}
  \item \textbf{Feature extraction}:
    \begin{itemize}
    \item Comment \'evalue-t-on qu'un \emph{predictor} est pertinent ou pas?
    \item Quels sont les bonnes strat\'egies pour cr\'eer des features? Ça doit
      dépendre des domaines...
    \item Quand est-ce qu'on a trop de \emph{predictors}?
    \item Quels modèles sont plus/moins sensibles aux nombres de
      \emph{predictors} qui leurs sont pr\'esent\'es?

      En d'autres mots est-ce que le modèle est capable de quelconque
      mani\`ere de s\'electionner les \emph{predictors} pertinents?
      
      Est-ce que le modèle aura tendance \`a overfitter ou mal g\'en\'eraliser
      avec trop de \emph{predictors}?
    \end{itemize}
  \item \textbf{Model Selection}:
    \begin{itemize}
      \item Quel est la bonne fonction de perte pertinente \`a \'evaluer?
      \item Y-a-t-il plus qu'une m\'etrique \'a \'evaluer?
        \emph{e.g} \emph{Recall} vs \emph{Precision}, etc.
      \item \emph{Validation Set} vs \emph{Test Set}:
        
        Dans la plupart des stratégies présentées, on fait une
        \emph{cross-validation} sur le \emph{training set} pour optimiser les
        hyperparam\'etres d'un modèle. L'optimisation des hyperparam\'etres est
        une forme d'entrainement au même titre que l'entraînement normal d'un
        modèle.

        Lors de cette \'etape, on peut estimer empiriquement la moyenne et la
        variance de sa fonction de perte, sur la du \emph{fold} réservée à la
        validation. Mais ces estimations sont généralement biaisées vers le bas,
        comme toute estimation issue d'un entraînement / fitting.

        Lorsqu'on résout un problème de régression, on peut évaluer la
        performance finale du modèle via un \emph{Test Set}. On obtiendra un
        estimation empirique de de la moyenne et la variance de la fonction de perte.

     
      \item \underline{Inner/Outer cross-validation}:

        Lorsqu'on a la puissance de calcul/le temps pour le faire on peut appliquer une
        stratégie de \emph{cross-validation} aussi pour le \emph{Test Set}.

        Donc au lieu d'appliquer une stratégie \emph{Hold-Out} en gardant un
        seul et unique \emph{Test Set} final, on applique aussi un \emph{K-Fold
          cross-validation} pour le \emph{Test Set}.

        \emph{Inner} pour la \emph{cross-validation} de chaque modèle.

        \emph{Outer} pour la \emph{cross-validation} l'évaluation du modèle final.

        Normalement si notre modèle généralise bien, il ressortira gagnant à
        chaque étape de la \emph{Outer cross-validation}
    \end{itemize}
    
\end{enumerate}


\end{document}